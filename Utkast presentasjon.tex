

Presentasjon

Innledning:

- Nytten av et bakkestasjonsnettverk
- Hvordan fungerer et bakkestasjonsnettverk
- Hvilke nettverk som er i dag, undersøkt de ulike nettverkene, valgte Carpcomm (hvorfor)
- Hva er målet vårt? Problemstilling: Koble oss opp mot Carpcomm space network ved hjelp av raspberry pi.

- Valg av raspberry pi:
		- billig
		- enkel for sitt bruk
		- liten
		- gpio pinne, muligheten til å kobles på criel port.

Prosjektet: Hva er gjort

- Sette opp raspberrry, installere programvaren til Carpcomm.
- Software:		
	mulighet for å motta signaler men ikke sende
	muligheten for å laste opp data på et delt nettverk
- Lage antenneovergang, sender signaler til antennestyringsboksen
- Installere hamlib, en programvare for å kunne styre rotoren til antennen.
- Lagde en overgang fra raspberry til antennestyringsboksen for å kunne styre antennen. 
- Utfordring: Antennen har vært nede så har ikke fått testen opp mot denne. Koblet oss til en tv motaker for å teste om vi kunne motta signaler. Dette fungerte :)

- Veien videre/konklusjon:

- Carpcomm mottar kun det satelitten sender og sender ingenting til satelitten. 
- Klart å motta signaler med den radioen vi har testet med, men ikke fått testet opp mot sattelitten. 
- Vi tror at Carpcomm er godt alternativ til et bakkestasjonsnettverk. 
- Lages en ny overgang mellom raspsperry pi og antennestyringsboksen eller å inkludere funksjonalliteten til antennestyringen inne i raspberry pi. 
- For at Raspberry pi skal klare å kommunisere med radioen så må vi lage en egen programvare. Dette er noe som kan gjøres som videreføring av prosjektet. Det





