\chapter{Ground Station Networks}
%This chapter will present the different technologies, theory behind the reasoning for having multiple ground stations listen to our satellite and calculations on sufficient power supply. 

This chapter will present some theory and simulations regarding ground station networks and power supply and information about some of the available technologies are explored. The reason why we chose a particular network, our implementation of it and our results are also presented. 

%Before we started our project, we did quite a lot of prestudy, covering which technology was available, to determine which fit our project best. This chapter will contain a summary of the different technologies, and also some theory behind the reasoning for having multiple ground stations listen to our satellite.

\section{Communicating with the satellite}

\begin{figure}
  \begin{center}
    \includegraphics[width=1.0\textwidth]{Figures/accesstid450_500_650horizont}
  \end{center}
  \caption[LOS for 450 and 650]{Time above the horizon for a satellite with altitude $h=450km$, $h=500km$ and $h=650km$ as a function of ground station latitude}
  \label{fig:access_horizon}
\end{figure}

\begin{figure}
  \begin{center}
    \includegraphics[trim = 5mm 30mm 5mm 0mm, clip, width=0.8\textwidth]{Figures/groundstation_satellite_geometry}
  \end{center}
  \caption[Ground station satellite geometry]{Illustration of geometry between a ground station and a satellite}
  \label{fig:ground_station_satellite_geometry}
\end{figure}

Since we don't know when the satellite will be launched we don't know the exact orbit. The project manager, Roger Birkeland, told us that we can assume the orbit will  have a height above the Earth somewhere between 450km and 650km and inclination of 98 degrees.

The height above the earth dramatically changes the time the satellite is seen by the ground station, see Fig. \ref{fig:access_horizon}. The signal will be weaker when the satellite is further away, it is therefore necessary to increase the minimum elevation angle $\epsilon$, see Fig. \ref{fig:ground_station_satellite_geometry}, to maintain SNR.  Previous work\cite{antennemaster,bildemaster} has calculated the minimum elevation angle, for a ground station here in Trondheim, for satellites with different altitudes and found that these effects cancel each other out. In the following we'll assume that the altitude is 500 km and the minimum elevation angle is 28 degrees.

\begin{figure}
  \begin{center}
    \includegraphics[width=0.8\textwidth]{Figures/ntnu_footprint}
  \end{center}
  \caption[NTNU footprint]{NTNU ground station range}
  \label{fig:ntnu_range}
\end{figure}

The result of this is that the ground station can communicate with the satellite whenever the ground track is inside a rough circle centered on the ground station, see Fig. \ref{fig:ntnu_range} for the estimated "range" of the ground station at Gløshaugen operating with these constraints. The efficiency of a network of ground stations is reduced when the ranges overlap, so to have a sufficient network the nodes must be geographically far apart. In this case the ground stations must be more than 1600 km apart to have maximum efficiency.

The Gløshaugen ground station will on average be able to communicate with the satellite 520 seconds per day. With a bit rate of 9600 bps 600kB can be downloaded per day on average.

\subsection{Other ground stations}
\begin{figure}
  \begin{center}
    \includegraphics[width=0.9\textwidth]{Figures/verdenskart}
  \end{center}
  \caption[world]{Map of the World}
  \label{fig:world}
\end{figure}

Fig. \ref{fig:access_horizon} shows that the access time for a (near) polar satellite is almost latitude independent for ground stations at latitudes below 45 degrees. The access times of a ground station those low latitudes are about about 280 seconds/day, i.e. half the duration we get here in Trondheim. For higher latitudes the access time increases dramatically. And the average access time for a ground station in Longyearbyen (78N) is in fact as high as 1300 seconds/day.

\section{Solar Power}

Communication with the satellite through the groundstation network demands power from the satellite. Without any electrical power a satellite will not be able to support its payload or radio communication. NUTS double CubeSat uses sunlight as an energy source through solar panels. 

Solar panels base their operation on the ability to convert sunlight into electricity. By using semiconductors the photovoltaic effect can be exploited. The conversion process where the suns radiation is converted into an electical current is achieved by creating mobile charged particles in the semiconductor. They are in turn separated by the device structure and produce the electrical current.(kilde) 

Batteries are used to store power during eclipse and support the payload. When the satellite is in eclipse the earth blocks the solar radiation and the battery must supply the power.

Identifying the satellites communication necessary to establish if a groundstation network is profitable. Forthcoming calculations are based on an estimate done by De Bruyne \cite{Satellite Power Systems}.

When estimating the period of the satellite the Earth and satellite orbit is assumed to be spheres. Kepler`s third law for circular orbits is used:

\begin{equation}
\left(\frac{2\pi}{T}\right)^2 = \frac{GM}{R^3}
\label{eq:Keplers_3}
\end{equation}

Where T is the period of the satellite, G is the gravitational constant\footnote{$G=\unitfrac[6.6742\cdot 10^{-11}]{km^3}{s^2}$}, M is the mass of the Earth\footnote{$M=\unit[5.9736\cdot 10^{24}]{kg}$} and R is the distance between the centers of mass of the Earth and the satellite.

\begin{equation}
T = 2\pi \left(\frac{R^3}{GM}\right)^{\frac{1}{2}}
\label{eq:satellite_period}
\end{equation}

where $R = R_\Earth + h$ where $R_\Earth$ is the radius of the earth\footnote{$R_\Earth = \unit[6371]{km}$} and h is the altitude of the satellite. 
Since it is uncertain what altitude the satellite will settle in after it is launched, two heights are used in the calculations and it's assumed that the satellite wil settle somewhere within interval $h_1=\unit[450]{km}$ and $h_2=\unit[650]{km}$. 
Which gives

\begin{equation}
T(h_1) = 5610s
\end{equation}

\begin{equation}
T(h_2) = 5850s
\end{equation}

when inserted into \autoref{eq:satellite_period}.

From De Bruyns equations \cite{Satellite Power Systems} the longest possible time in eclipse can be calculated. The worst case average power is approximately $P_{avg} = 5.42 W$ from by De Bryens calculations. This power is claculated when the satellite has its longest time in eclipse.

\begin{equation}
	t_{ecl,max} = 2R_{sat}\left(\frac{R_{sat}}{R\times M}\right)^{\frac{1}{2}}
	\label{Maximum time in eclipse}
\end{equation}

\begin{equation}
	P_{avg,orbit} = P_{avg}\times\frac{(t-t_{ecl,max})}{t}
	\label{Average effect during one orbit}
\end{equation}

Putting in $h_1 =\unit[450]{km}$ and $h_2 = \unit[650]{km}$ gives:

\begin{equation}
t_{ecl,max_1} = 2151.1 s = 35.9 min
\end{equation}

\begin{equation}
P_{avg,orbit_1} = 3.34W
\end{equation}

\begin{equation}
t_{ecl,max_2} = 2118.9 s = 35.3 min
\end{equation}

\begin{equation}
P_{avg,orbit_2} = 3.46W
\end{equation}

These are simplified calculations where the temperature changes of the solar cells is not taken into account. As the satellite moves through orbit the temperature will effect the solar cells, but including this effect requires more extensive calculations. The average power calculated here is based on worst-case estimate from De Bruyns \cite{Satellite Power Systems}, the real average power can be calculated with the exact orbit parameters. 

The battery used in the NUTS Cubsat wil consist of lithium-ferrite-phosphate cells $(LiFePO_4)$\cite{Overview of NUTS}. 
These cells have a typical voltage of 3.3 V. NUTS wil have $4\times \unit[1.1]{Ah}$ cells, where two is in serie and two is in parallell(parallell i serie eller serie i parallell?). This means a total of $C=\unit[2.2]{Ah}$ at $\unit[6.6]{V}$ (\cite{Satellite Power Systems} unødvendig?). This is used to calculate the worst-case Depth Of Discharge (DOD). The DOD represents the percentage of the discharged battery capacity expressed as a percentage of maximum capacity. It indicates the state of charge where $100\% $= empty and  $0\%$ = full (KILDE).

The battery capacity used during eclipse: 
\begin{equation}
C_{ecl} = \frac{P_{avg,orbit}}{V_tot}\times t_{ecl,max}
\label{Battery capacity}
\end{equation}

 \begin{equation}
DOD_{max} = \frac{C_{ecl}}{C_{tot}}
\label{Depth Of Discharge}
\end{equation}

Calculated with the different heights gives:

equation \ref{Battery capacity}
$C_{ecl_1} = \frac{3.34W}{6.6V}\times(\frac{35.9min}{60\frac{min}{hour}}) = 0.3Ah$

equation \ref{Depth Of Discharge}
$DOD_1 = 0.136 = 13.6 \%$

equation \ref{Battery capacity}
$C_{ecl_2} = \frac{3.46W}{6.6V}\times(\frac{35.3min}{60\frac{min}{hour}}) = 0.31 Ah$

equation \ref{Depth Of Discharge}
$DOD_2 = 0.141 = 14.1\%$

The battery capacity will decrease with the number of charge-discharge cycles. Discharge of at least 80\% is refered to as deep discharge. When the critical DOD is reached it will be in risk of battery failure. From our calcultations the DOD is not in the critical region. 

\vspace{5 mm}To establish the communication time a simplified power balance is used. 

\vspace{5 mm}$P_{radio} = 2.5 W$

$P_{payload} =$ $5 W$  in  $90s$

Microcontroller is in standby and with 18.5mA and 3.3V it uses:

$P_{standby} = 0.0612W$ 

These numbers are supplied from NUTS and are rough estimates of what can be expected.

\begin{equation}
P_{avg,orbit}\times t - P_{payload}\times 90 s - P_{standby}\times t= P_{radio}\times t_{communication}
\label{eq:communication_time}
\end{equation}

Inserting $h_1$ and $h_2$ into \autoref{eq:communication_time} gives:

\begin{equation}
t_{communication} = \unit[7172.12]{s} =\unit[110.5]{min}
\end{equation}

\begin{equation}
t_{communication} = \unit[7778.77]{s} =\unit[129.6]{min}
\end{equation}

The calculations on power- and orbit estimates results in long communication time. There are several simplifications done during this calculations that must be taken into account. The power drawn from the satellite are rough estimates and the available communication time will likely differ from the these results. The calculated communication time provides a good indication and it can be seen that it is time available to connected to a groundstation network.

\subsection{Ground station networks}

\subsubsection{NTNU and Aalborg University}

\begin{figure}
  \begin{center}
    \includegraphics[width=0.7\textwidth]{Figures/range_ntnu_aalborg}
  \end{center}
  \caption[NTNU Aalborg]{Ground station network: NTNU and Aalborg}
  \label{fig:range_ntnu_aalborg}
\end{figure}

A ground station network consisting of NTNU and Aalborg University is not very efficent, see Fig. \ref{fig:range_ntnu_aalborg}.

\subsubsection{NTNU and UNIS}
 
\begin{figure}
  \begin{center}
    \includegraphics[width=0.7\textwidth]{Figures/range_ntnu_svalbard}
  \end{center}
  \caption[NTNU Aalborg]{Ground station network: NTNU and UNIS}
  \label{fig:range_ntnu_unis}
\end{figure}
\section{Network technology for ground stations}

\subsection{Genso}
information about Genso
\subsection{PYXIS}
The BlueBox is part of a distributed ground station network called PYXIS, developed primarily for the AAUSAT3 by Aalborg University (2013 \cite{aausat3}). The PYXIS goal is to offer a robust and effective ground station network for satellite developers, and one of the key factors is that everyone is free to setup a ground station using the open source BlueBox hardware. 

The PYXIS concept includes a backend server, BlueBox hardware and a Ground Station Server (GSS). The back-end server runs an individual instance for each satellite utilizing the BlueBox, and is operated by the persons responsible for the ground station. 

The BlueBox itself is hardware to receive and transmit signals from the satellites.

Control of the BlueBox and ground station mechanics is handled by the GSS, and both the BlueBox and the GSS is operated by the responsible for the Ground station. 

Both the backend server and the GSS is already in place at each ground station, and to join the PYXIS network we would only have to make a BlueBox, and test that it works.

%\cite{aausat3}
\subsection{Carpcomm Space Network}
Carpcomm is a private company that delivers a plug and play ground station \cite{carpcomm-gs1} that costs $700. The software for the ground station is open source and is provided pre.compiled for x86 and arm debian.
It is compatible with the Carpcomm Space Network \cite{carpcomm-sn}. 
The advantage of using this solution is that the network is actually functioning, though there are few other operational ground stations.

\paragraph{Why we chose Carpcomm}
After exploring the different alternatives, BlueBox seemed to be a good choice. However this would require some support from Aalborg University. This was impossible, as they were busy with a satellite launch of their own. GENSO is not up and running yet, so it is not suitable for our purposes. The Carpcomm Space Network is a network of ground stations that already have participants. This network is actually up and running and all the necessary resources are available at their web site. We wanted to test this ground station network without interfering with the system already in place. The solution was to use a Raspberry pi to connect the ground station at Gløshaugen to the Carpcomm Space Network.
 
\section{Raspberry pi}
Raspberry Pi is a small computer, with everything gathered in one board. In our project we will be using the B model, revision 2, which have a 700MHz ARM CPU, 512MB of RAM and a SD-card reader, in addition to the leads to connect to different devices, for  the full overview, see figure \ref{fig:raspberrypihighlevel}. The recommended operating system is Raspbian, a linux distribution based on Debian. The Carpcomm home page lists Raspbian as one of its supported platforms.

%The Raspberry pi was originally intended to help teach programming, but it can also perform many of the standard computer tasks, and it can be connected to a monitor or tv using an HDMI lead. In our project we hope to be able to use a Raspberry pi to run the software required to control the ground stations. 
The software provided by the Carpcomm project has Raspbian as one of its supported platforms, so we hope this will work well.

\begin{figure}
	\begin{center}
		\includegraphics[width=0.7\textwidth]{Figures/raspberrypi_modelb_hl.jpg}
	\end{center}
	\caption[Raspberry pi highlevel]{A highlevel schemantic of the Raspberry pi, model B rev 2}
	\label{fig:raspberrypihighlevel}
\end{figure}

To control the movement of the antennas, a serial port is needed. Raspberry Pi has an serial port included in its gpio (general purpose input/output) connector. This serial port uses ttl-standard for its voltage levels, this is 0/3.3V while RS232 which is the standard used in computers uses (3V-15V)/-(3V-15V). Because of this an converter is needed. A custom circuit board was made using the MAX3232 RS232 line driver.%We chose to make an custom circuit board using the MAX3232 RS232 line driver. 
%The circuit board is designed to be mounted on the gpio connector, because of small space in the case for the Raspberry Pi, the output is connected with a cable to the external connector.
The circuit board is mounted on the gpio connector and, because of the small case, a cable is used to connect to the external connector, see \autoref{kabeln}.

\begin{figure}
	\begin{center}
		\includegraphics[width=0.7\textwidth, trim=360 220 190 180, clip=true]{../Schematics/RPI-RS232-schematic.pdf}
	\end{center}
	\caption{Schematics for the RS232-converter}
	\label{fig:UART-RS232}
\end{figure}

\begin{figure}
\begin{subfigure}{.5\textwidth}
	\centering
	\includegraphics[width=\textwidth]{Figures/rasp-1}
	\label{fig:pien}
\end{subfigure}
\begin{subfigure}{.5\textwidth}
	\centering
	\includegraphics[width=\textwidth]{Figures/rasp-2}
	\label{fig:kabeln}
\end{subfigure}
\caption{Our finished product}
\end{figure}


\section {Connecting to the Network}

To connect the ground station to the Carpcomm Space Network the Carpcomm software was installed on the Raspberry pi. The CarpSD ground station control software used is an open-source program primarily developed by the team behind Carpcomm. The purpose of this software is to make it possible to connect to the Carpcomm Space Network server. The software runs as a background process and makes it possible to control the ground station if you have access to a internet browser. The software runs continuously in the background making it possible to connect remotely at any time and start receiving data from satellites.

To register our ground station with the network an entry for our station, plotting the latitude, longitude and elevation, were created. 
Before starting testing, the software controlling the antenna had to be installed. The Hamlib rotator library was chosen since the ground station already uses it and CarpSD supports it. 

With everything set the testing of the station could start. The testing was started of with connecting to a small TV-antenna to register the signal. On the station page at the Carpcomm website the frequency was set to 433.3 MHz. This frequency was chosen because it is used by the police in Trondheim, and there are likely to be a lot of transmissions there. Data (see \autoref{fig:Transmission}) was successfully received and sent to the Carpcomm Space Network. After receiving data from the TV-antenna, a full-scale antenna was used.

\begin{figure}
	\centering
	\includegraphics[width=0.8\textwidth]{Figures/sattelite_transmition}
	\label{fig:Transmission}
	\caption{A signal recorded by the Carpcomm Space Network}
The x-axis is the frequency and the y-axis is time. Black and gray is noise and the white line to the left of the yellow line is the signal.
\end{figure}

%The Carpcomm software succeeded in receiving data, but the automated rotor control did not work. This is a problem that remains unsolved and must be worked on further. Another problem is that the Carpcomm software only supports receiving data, not sending. This means that one must command the satellite to start transmitting, and some other solution must consequently be devised. 


