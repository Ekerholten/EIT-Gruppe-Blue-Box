\section{Summary}
The NTNU Test Satellite is using amateur radio and the transfer rate will therefore be very limited, of only 9600 bps. Combined with the limited access time to the satellite this means a very limited 600 kB of data can be downloaded per day. 

Different ground station network are today available, but a network is dependent on other ground stations as participants to be beneficial. Carpcomm Space Network is an existing network which is already connected, and therefore seemingly the best choice. 

To connect the ground station at Gløshaugen to the Carpcomm Space Network a Raspberry Pi were used to run the required software. During our work we discovered that it lacks support for many kinds of radios and that the rotor control functionality is still experimental. In addition it cannot transmit to the satellite, and can not be used to initiate downloads. 

Results from our simulations show that even a small network with Norwegian/Scandinavian universities will result in a significant gain in download capacity. If the weaknesses in the technology can be overcome the improvements can be substantial. Another factor that must be taken into considerations when planning a network of ground stations is the position of the different nodes. The efficiency of a network of ground stations is reduced when the communication ranges overlap. The range of the Gløshaugen ground station covers much of Northern Europe, so efficiency will be reduced to a level more in line with lower latitude stations. A network of ground stations demands more power as the communication time increase. Calculations show that there is enough power available to meet these increased demands.

However, further work must be carried out so that the Carpcomm Space Network can be fully exploited, but if this is done it will bring great benefits.