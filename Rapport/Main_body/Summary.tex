\section{Summary}
The NTNU Test Satellite is using amateur radio and the transfer rate will therefore be very limited, of only 9600 bps. Combined with the limited access time to the satellite this means a very limited 600 kB of data can be downloaded per day. 

Different ground station network are today available, but a network is dependent on other ground stations as participants to be beneficial. Carpcomm Space Network is an existing network which is already connected, and therefore seemingly the best choice. 

To connect the ground station at Gløshaugen to the Carpcomm Space Network a Raspberry Pi were used to run the required software. During our work we discovered that it lacks support for many kinds of radios and that the rotor control functionality is still experimental. In addition it cannot transmit to the satellite, and can not be used to initiate downloads. 

Results from our simulations show that even a small network with Norwegian/Scandinavian universities will result in a significant gain in download capacity. If the weaknesses in the technology can be overcome the improvements can be substantial. Another factor that must be taken into considerations when planning a network of ground stations is the position of the different nodes. The efficiency of a network of ground stations is reduced when the communication ranges overlap. The range of the Gløshaugen ground station covers much of Northern Europe, so efficiency will be reduced to a level more in line with lower latitude stations. A network of ground stations demands more power as the communication time increase. Calculations show that there is enough power available to meet these increased demands.

However, further work must be carried out so that the Carpcomm Space Network can be fully exploited, but if this is done it will bring great benefits.

\section{Further Work}
As it stands now the Carpcomm Space Network can not be used at the same time as the rest of the ground station systems. It is fairly simple to implement code to upload data to Carpcomm, so a possible solution may be to integrate CSN support in the already existing software. Then the existing solution can be used to control the radio and antennas, and data can be uploaded to the network when the ground station is idle.

Alternatively the Carpcomm software can be used. In that case rotor control must be improved. A modification to CarpSD or the driver for the rotor interface will probably be necessary. The existing rotor interface has some quirks that makes the movement very choppy, which can be bad for the antennas and rotors. Because of this it is desirable with a new rotor interface. The Raspberry Pi have the necessary inputs and outputs to act as an interface, it just needs some voltage converters and software to control it.

The solution using the gpio connectors to connect to the rotor control must be improved. Either a more robust cable and connection must be made, or the serial port connector integrated in the chassis.
\section{Social benefits}


Carpcomm is a ground station network that connects ground stations all over the world. This makes it possible for ground stations with short or rare satellite passes to get more communication time. Ground stations with broader communication time can obtain data from other satellites and provide it to the ground station in need of this information. The rules for accessing the Carpcomm space network are: for every byte of data you upload, you can download one byte of data from your satellite. This provides a give and take society, and it is important that each ground station has the opportunity to provide data to others. 

For ordinary people, it might not be easy to see the great social benefits of a ground station network. As mentioned, a ground station network will increase the number of downstream per pass. This is profitable for Norway since we have a landscape with a lot of mountains, which provides limitations on communication with satellites. With more data available, one can more easily investigate and improve different aspects of a satellite mission like the satellite launch, satellite power system and payload. This will improve the usefulness of satellite projects, such as NUTS. More data will provide greater benefits for those working with smaller satellites, like for example students. Increased knowledge is beneficial for education, and this is very useful to the society. 

When connecting to a network all the participants will benefit, in addition to getting more date to your own project other will receive more data as well. Small-scale satellite projects created by countries with limited resources will gain from joining a network of ground stations. B creating a network of ground stations one will achieve collaboration between countries, university and people. 
Sharing of recourses will give more knowledge to a wider range of people and increase the likelihood of new projects being created. 

