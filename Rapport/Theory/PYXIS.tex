\subsection{PYXIS}
The BlueBox is part of a distributed ground station network called PYXIS, developed primarily for the AAUSAT3 by Aalborg University (2013 \cite{aausat3}). The PYXIS goal is to offer a robust and effective ground station network for satellite developers, and one of the key factors is that everyone is free to setup a ground station using the open source BlueBox hardware. 

The PYXIS concept includes a backend server, BlueBox hardware and a Ground Station Server (GSS). The backend server runs an individual instance for each satellite utilizing the BlueBox, and is operated by the persons responsible for the ground station. 

The BlueBox itself is hardware to recieve and transmit signlas from the satellites.

Control of the BlueBox and ground station mechanics is handled by the GSS, and both the BlueBox and the GSS is operated by the responsible for the Ground station. 

Both the backend server and the GSS is already in place at each ground station, and to join the PYXIS network we would only have to make a BlueBox, and test that it works.