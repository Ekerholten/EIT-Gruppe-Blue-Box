\chapter{Introduction}
\label{chap:introduction}

The students here at NTNU will be building and launching a double Cubesat in the near future. It will have a scientific payload consisting of a infrared camera, which will be used to study gravity waves in the atmosphere. One problem is that the transfer rate for a satellite using amateur radio is low and the number of passes are limited. We wanted to look more into this problem and find a solution. The download capacity is limited by the amateur radio protocol, antennae and the amount of time the satellite is above the horizon. We don't want to change the former, but can increase the latter by joining a ground station network.

%Since the transfer rate is limited by the antenna its hard to increase the transfer rate without changing the antenna itself. We therefore chose to dive more into how we could increase the number of downstream per pas. We looked into how we could utilize a network of ground stations. We needed to find an existing network that had other ground stations already connected. Carpcomm had seemingly a well established network and was one of the main reason for why we chose it.

%\section{NUTS}

%NUTS is a student driven satellite project at NTNU. The goal of the project is to design and build a double Cubesat. The student satellite project is organized under the  Department of Electronics and Telecommunications. 

\section{The Project group}
The project group consisted of 6 people from 5 different institutes, so we have competency in a variety of fields. The group members can be seen in \autoref{tab:groupmembers}.

\begin{table}
	\begin{center}
		\begin{tabular}{|l|l|}   
			\hline      
			\bf{Name} & \bf{Background} \\ 
			\hline
			Marius Ekerholt & Computer technology\\     
			\hline
			Eirik Skjeggestad Dale & Computer technology\\     
			\hline
			Hanne Thorshaug Andresen & Energy and Environmental technology\\     
			\hline
			Leif-Einar H. Pettersen & Electronics and Telecommunications\\     
			\hline
			Børge Irgens & Theoretical Physics\\     
			\hline
			Hallstein Skjølsvik & Electronics and digital design\\     
			\hline
		 \end{tabular}
	\end{center}
	\caption{Group members}
	\label{tab:groupmembers}
\end{table}
