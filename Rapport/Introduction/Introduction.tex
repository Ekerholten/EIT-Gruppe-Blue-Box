\chapter{Introduction}
\label{chap:introduction}
This chapter will contain a short introduction to our project. That means what we're doing and what people are working on the project. It will also be discussed some short background on why we're doing this project.
\section{The Project group}
The project group consisted of 6 people from 5 different institutes, so we have competency in a variety of fields. The group members can be seen in \autoref{tab:groupmembers}.

\begin{table}
	\begin{center}
		\begin{tabular}{|l|l|}   
			\hline      
			\bf{Name} & \bf{Background} \\ 
			\hline
			Marius Ekerholt & Computer technology\\     
			\hline
			Eirik Skjeggestad Dale & Computer technology\\     
			\hline
			Hanne Thorshaug Andresen & Energy and Environmental technology\\     
			\hline
			Leif-Einar H. Pettersen & Electronics and Telecommunications\\     
			\hline
			Børge Irgens & Theoretical Physics\\     
			\hline
			Hallstein Skjølsvik & Electronics and digital design\\     
			\hline
		 \end{tabular}
	\end{center}
	\caption{Group members}
	\label{tab:groupmembers}
\end{table}

\section{Network}

The problem today is that the transfer rate for a satelite using amateur radio is low and the the number of pases are limited. We wanted to look more into this problem and find a solution. Since the transfer rate is limited by the antenna its hard to increase the transferrate without changing the antenna itself. We therefore chose to dive more into how we could increase the number of downstream per pas. We looked into how we could utilize a network of ground stations. We needed to find an existing network that had other ground stations already connected. Carpcomm had seemingly a well established network and was one of the main reason for why we chose it.

\section{NUTS}

NUTS is a student driven satellite project at NTNU. The goal of the project is to design and build a double Cubesat. The student satellite project is organized under the  Department of Electronics and Telecommunications. 