\begin{abstract}
The NTNU Test Satellite will be using amateur radio and will therefore have a very limited transfer rate of 9600 bps. Combined with the limited access time to the satellite this means a very limited 600 kB of data can be downloaded per day. 

The aim of this project was to consider some possible networks and set up a ground station network at the student ground station here at Gløshaugen. 

We have set up a Raspberry Pi as a ground station and connected it to the Carpcomm Space Network.

\end{abstract}

\renewcommand{\abstractname}{Acknowledgements}
\begin{abstract}
We would like to thank the NUTS project leader Roger Birkeland and the "NUTS studentsatellitt" EiT- group for many good ideas and advice. Many thanks also to Knut Magnus Kvamtrø for help with connecting to the ground station.
\end{abstract}