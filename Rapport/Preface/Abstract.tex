\begin{abstract}

The NTNU Test Satellite will be using amateur radio and will therefore have a very limited transfer rate of 9600 bps. Combined with the limited access time to the satellite this means a very limited 600 kB of data can be downloaded per day. 

The aim of this project was set up a ground station network at the student ground station here at Gløshaugen and increase the download capacity. To achieve this several ground station network were explored and the Carpcomm Space Network was selected as the most suitable for this project. 

We have set up a Raspberry Pi as a ground station and connected it to the Carpcomm Space Network. However we've found that it lacks support for many kinds of radios and the rotor control functionality is still experimental. And in addition it can't transmit to the satellite, so it can't be used to initiate downloads. 

Our simulations show that even a small network will drastically increase the amount of data that can be downloaded. Calculations on the power supply for the satellite shows that there is sufficient power available to connect and communicate with a ground station network. If the challenges related to the rotor functionality can be overcome the improvements on data obtained from the satellite can be substantial.


\end{abstract}

\renewcommand{\abstractname}{Preface}
\begin{abstract}
This report is a part of the result of the Experts in Teamwork village TFE4850 Student Satellite, spring 2013. EiT is a interdisciplinary course, where students from different study programs work together to achieve a common goal. 

We would like to thank the NUTS project leader Roger Birkeland, the facilitators and all the other village groups for many good ideas and advice. Many thanks also to Knut Magnus Kvamtrø for help with connecting to the ground station.
\end{abstract}