Midtveispresentasjon.

Innhold:
- Endring fra bluebox/Pyxis til raspberry/carpcomm
- Forklaring rundt carpcomm/raspberry
- Hva vi har gjort:
	- Ferdig krets, mangler å bytte en kontakt. (serial)
	- Sett på simulering av enkeltstasjoners nedlastningskapasitet gjennom AGI STK
	- Satt opp Linux og carpcomms nettverksprogram på raspberry
- Utfordringer:
	- Avhengighet av andre.
		Vi har vært ganske avhengig av støtte fra andre både når det kom til blue-box og carpcomm. Vi skal bruke deres antenner og systemer så vi er avhengig av sammarbeid med de. Vi tror at dette vil bedre seg videre utover i prosjektet og at vi blir mer og mer selvstendige. 
	- liten boks, vanskelig å få satt på plass kontakt og robusthet. Vi må vurdere en ny løsning med antenne styre tilkoblingen på rasberry. den er litt lite robbust. skulle kanskje vært montert fast på kortet. problemet er bare at det er dårlig med plass så vi får se om vi kommer opp med en god permanenet løsning.
	- Prosjektet kan være litt vannskelig med tanke på parrarellisering. kan til tider være vannskelig å fordele arbeid siden mye går på enten raspberryn eller på AGI STK. Vi prøver å fylle tiden med litt refleksjoner og fokus på gruppearbeid :P 
	 



- Hva skal vi gjøre:
	- Sette opp boksen med antennestyringsprogrammer (koble til bakkestasjon), og koble til nettet
	- Teste at vi får tatt imot data og lastet disse opp til nettet.
	- Batterilevetidsanalyse
	- Analysere nedlastningskapasitet i nettverk av bakkestasjoner.

HVEM SIER HVA:

----
-Innledning: Hanne
----
-Hva vi har gjort: Hallstein
----
-Utfordringer: Marius
----
-Hva skal vi gjøre: Eirik