Da vi regner med at vi ikke kommer til å få tid til å bygge den nye versjonen av Blue box, vil vi satse på å sette sammen den gamle versjonen med modifikasjon. Med denne ønsker vi å motta data fra satelitt, men ikke sende. 

Dette trenger vi fra Aalborg universitet:
-CAD: Utlegg av kretskort, slik at vi kan etse kretskortet selv.
-Komponentliste: Trenger en komplett liste over komponenter som ble brukt under byggingen av den første versjonen av Blue box. Eventuelt komponentene til den modifiserte versjonen (Med ny forsterker). 
-Informasjon om den modifiserte versjonen av Bluebox (Den versjonen med ny effektforsterker).
-Informasjon om den nye versjonen som er under utvikling: Er det store forandringer på utlegg? Eller er det bare bytte av noen komponenter. Er de to versjonene kompatible, både software og hardware messig?
-Utdypende dokumentasjon av funksjonene til Blue box. Dersom det er skrevet noen rapporter, datablad eller tilsvarende om Blue box, ville dette være til stor hjelp for oss. Med å lese dette kan vi få mer kunnskap om den teoretiske bakgrunnen til prosjektet. 
-Protokolldefinisjoner: Dersom vi ønsker å sette opp et nettverk med Aalborg universitet, vil det være greit å vite hvordan nettverksprotokollen ser ut. 