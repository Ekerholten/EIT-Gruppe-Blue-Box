Prosjektdefinisjon Eksperter i team: BlueBox

En BlueBox er en USBdrevet bakkestasjon, utviklet for AAUSAT3 cubesat fra Aalborg Universitet, for å motta og sende signaler fra amatørsatellitter. Hver bakkestasjon trenger en BlueBox for å kommunisere med satellitter.  For å oppnå flere kommunikasjonsvinduer for satellitter er det ønskelig å lage et nettverk av bakkestasjoner. PYXIS er et slikt nettverk bestående av flere bakkestasjoner, utviklet av studenter ved Aalborg Universitet.
  
Vårt mål under arbeidet i Eksperter i team er å lage en BlueBox og teste at denne klarer å motta signaler. Det er ønskelig at dette prosjektet kan legge til rette for at vår bakkestasjon kan kobles opp mot nettverket PYXIS. 

Vi ser for oss at alle skal kunne bidra med sin kompetanse under arbeidet med dette prosjektet.  En del av gruppen sitter inne med kompetanse om programmering noe som vil brukes i softwaredelen av prosjektet.  Det er også nødvendig å undersøke om et nettverk av bakkestasjoner kan løse nåværende kommunikasjonsutfordringer og hvordan dette skal gjøres. Den teoretiske delen er svært viktig og kan gjøres uavhengig av faglig bakgrunn. Vi har også praktiske oppgaver som å lage og teste kretskort, ferdigstille BlueBoxen og teste at den virker.