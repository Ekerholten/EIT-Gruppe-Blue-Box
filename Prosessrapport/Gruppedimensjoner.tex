Resultater fra prosessøvelse: Gruppedimensjoner.

Gruppen er produktiv ut ifra sitt formål: Gruppen er trolig ikke på det modenhetsnivået som denne typen oppgave krever. Det er mulig at oppgaven krever en gruppe på nivå 3 eller 4 (control og opposition), mens vår gruppe er på et lavere nivå. Med tanke på at ingen i gruppen scoret oss høyt på punkt 5 (gruppen er preget av oppsoisjon... ), er det tydelig at vi ikke stiller hverandre kritiske spørsmål. Vi mangler trolig gruppegenskapen "opposition" og er derfor ikke på modenhetsnivå 4 (Inovation). 
	Det er mulig at gruppen er på modenhetsnivå 1 (Reservation), med tanke på at gruppen scoret høyt på punkt 3 (...det er en hyggelig og trivelig gruppe å være med i). Noe som indikerer at vi innehar mye av egenskapen "Nurture". Mer sansynelig er det av gruppen er på nivå 2 (Team spirit). Da gruppedeltagerene er i stand til å innordne seg etter andre, noe som kom fram under prosessøvelse "Roller" og den påfølgende diskusjonen. I tillegg er gruppedeltakerene oppdelt i spesifikke roller som overlapper hverandre i mindre grad.    
 
	
En annen mulighet er at oppgaven krever en gruppe med modenhetsnivå 1 eller 2. Oppgaven er delt opp i flere forskjellige områder (HW, SW, simulering og energiberegninger), og gruppen er avhengig av eksterne resurser (Aalborg, elektronikklab-bemanning og NUTS personell). Av den grunn kan det tenkes at gruppen, og eksterne burde vært kooridnert av en overordnet autoritet, noe som er typisk for modenhetsnivå 1 og 2. 
	Siden gruppen har valgt å ha en flat ledelsestruktur kan dette bety at modenheten i gruppen er på nivå 3. Dette kan også føre til at gruppen arbeider lite produktivt ut i fra sitt formål.
	
	 
En tredje mulighet, som kanskje er mer sannsynelig enn de to oversrående er at gruppens modenhetsnivå ligger på 2. Men gruppen prøver å inkorporere en ledelsestruktur som tilsvarer et modenhetsnivå på 3 eller høyere, samtidig som at oppgaven som skal løses krever et like høyt modenhetsnivå. 

Siden alle i gruppen er enige om at dette er en hyggelig gruppe å være med i, indikerer det at vi er godt innenfor vår komfortsone. Da man må gå utenfor sin komfortsone for å raskere øke gruppens modehnhet, er det realistisk å tro at gruppen ikke har kommet høyere enn nivå 2. 