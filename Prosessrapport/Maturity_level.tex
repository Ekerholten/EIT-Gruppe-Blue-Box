\usepackage{comment}

Our group has completed several process exercises. The two exercises "Group Dimensions" (Gruppedimensjoner) and "Roles" (Roller) helped us to determine the  maturity of our group. In the exercise "Roles" we were to score each group members individual properties as a group participant. This was done by grading several statements according to how much we agreed that they fit a person. [Attachment]. The exercise "Group Dimensions" were similar, except this time we graded the groups properties as a whole. During the exercise "Group Dimension" an interesting discussion started at the second statement. 
\\
The second statement was: "...the group is productive compared to its purpose." Here the group answered differently. Some meant that the group was effective, while others, particularly Hallstein, meant that the group was somewhat ineffective. Hallstein used Scwarzes 4th ground rule for effective groups; Explain your reasoning and intent (Scwarz 2002). The reason Hallstein scored this group low on effectiveness compared to its purpose was because he had a suspicion that the groups leadership policy did not fit the group's task. He had a suspicion that there was some sort of mismatch between the groups maturity level and the maturity level needed for the task. In order to decide this, we tried to determine our own group maturity, and the maturity needed for the task.  
	Since the group is quite newly formed, it is reasonable to assume that the group has a low maturity. Probably level 1 or 2 (Reservation or Team Spirit). In "Group Dimensions" the group scored quite high on statement 3 (... hyggelig og trivelig), which indicates that the group has the property "Nurture". The process exercise "Roles" with the following discussion indicated that the group members were able to agree with each other and follow other member's lead. Therefore does the group probably have the property "Dependence". In "Group Dimensions" the group scored low on statement 1, which supports this. 
	In the exercise "Group Dimensions" the group scored high on statement 4 (...the work load is evenly distributed among the group members). This suggests that the group have the ability to control its work flow and distribution. The group also scored low on statement 6 (...the group shows little respect for regulations, to show up in time, keep appointments, prepare for or complete tasks effective and thoroughly). This can indicate that the group is structured and respect the authority of the other group members, which supports the view that group has the property "Control". Therefore we may assume that the maturity of the group is at level 3 (Production).
	At statement 5 (...the group is facing internal opposition, disagreement/ ill will) the group scored low. This shows that the group does not have the property "Opposition". This means that it is a nice group to participate in, which is reflected in statement 3, however the group has not yet evolved to level 4 (Innovation).
\\	
The maturity level needed for the task is a bit more difficult to decide. The task is quite wide. It covers several fields of studies; hardware, software, Energy and power calculations, radio communication, and simulations. That means that we each focus on a specific part of the task, and are very dependent of each other, since no one can cover others part. Therefore it is reasonable to assume that the group need a leader that is able to make everything fit together. Like the operation allegory in the note [note]. We may make the the conclusion that the task needs a maturity level of 1 or 2. Since the group is not making anything new and innovative, it does not require a level 4 group. The task is not directly a production oriented task either. The group is to produce one example, not a large quanta that requires a high level of control. 

As an additional point, the groups task demanded a lot of assistance from external resources. When these resources were unable to aid us, the entire project halted, and caused us to be very ineffective. It would have been a huge advantage if an external coordinator made sure we could interact with the external resources when we needed it. This supports that this task demand a level 2 leadership structure.     

         
\begin{comment}
Gruppen er produktiv ut ifra sitt formål: Gruppen er trolig ikke på det modenhetsnivået som denne typen oppgave krever. Det er mulig at oppgaven krever en gruppe på nivå 3 eller 4 (control og opposition), mens vår gruppe er på et lavere nivå. Med tanke på at ingen i gruppen scoret oss høyt på punkt 5 (gruppen er preget av oppsoisjon... ), er det tydelig at vi ikke stiller hverandre kritiske spørsmål. Vi mangler trolig gruppegenskapen "opposition" og er derfor ikke på modenhetsnivå 4 (Inovation). 
	Det er mulig at gruppen er på modenhetsnivå 1 (Reservation), med tanke på at gruppen scoret høyt på punkt 3 (...det er en hyggelig og trivelig gruppe å være med i). Noe som indikerer at vi innehar mye av egenskapen "Nurture". Mer sansynelig er det av gruppen er på nivå 2 (Team spirit). Da gruppedeltagerene er i stand til å innordne seg etter andre, noe som kom fram under prosessøvelse "Roller" og den påfølgende diskusjonen. I tillegg er gruppedeltakerene oppdelt i spesifikke roller som overlapper hverandre i mindre grad.    
 
	
En annen mulighet er at oppgaven krever en gruppe med modenhetsnivå 1 eller 2. Oppgaven er delt opp i flere forskjellige områder (HW, SW, simulering og energiberegninger), og gruppen er avhengig av eksterne resurser (Aalborg, elektronikklab-bemanning og NUTS personell). Av den grunn kan det tenkes at gruppen, og eksterne burde vært kooridnert av en overordnet autoritet, noe som er typisk for modenhetsnivå 1 og 2. 
	Siden gruppen har valgt å ha en flat ledelsestruktur kan dette bety at modenheten i gruppen er på nivå 3. Dette kan også føre til at gruppen arbeider lite produktivt ut i fra sitt formål.
	
	 
En tredje mulighet, som kanskje er mer sannsynelig enn de to oversrående er at gruppens modenhetsnivå ligger på 2. Men gruppen prøver å inkorporere en ledelsestruktur som tilsvarer et modenhetsnivå på 3 eller høyere, samtidig som at oppgaven som skal løses krever et like høyt modenhetsnivå. 

Siden alle i gruppen er enige om at dette er en hyggelig gruppe å være med i, indikerer det at vi er godt innenfor vår komfortsone. Da man må gå utenfor sin komfortsone for å raskere øke gruppens modehnhet, er det realistisk å tro at gruppen ikke har kommet høyere enn nivå 2. 
\end{comment}