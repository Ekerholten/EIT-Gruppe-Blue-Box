\section{April 3, Maturity level}
Our group has completed several process exercises. One of these was the exercise "Group Dimensions which helped us determine the maturity of the group. In this exercise we were to score the properties of the group. This was done by grading several statements according to how much we agreed that they fit our group. During the second statement, an interesting discussion occurred. 

The second statement was: "...the group is productive compared to its purpose." Here the group answered differently. Some meant that the group was effective, while others, particularly Hallstein, meant that the group was somewhat ineffective. Hallstein used Schwarz ground rule four; Explain your reasoning and intent \cite{EffectiveGroups}, which describe effective groups. The reason why Hallstein scored the group low on effectiveness compared to its purpose, was because he had a suspicion that the groups leadership policy did not fit the group's task. He had a suspicion that there was some sort of divergence between the group’s maturity level and the maturity level needed for the task. In order to decide this, we tried to determine our own group maturity, and the maturity needed for the task. 

Since the group is quite newly formed, it is reasonable to assume that the group has a low maturity. Probably level 1 or 2 (Reservation or Team Spirit). In "Group Dimensions" the group scored quite high on statement 3 (... this is a nice and comfortable group to be a part of), which indicates that the group has the property "Nurture". At the first statement (... the group is controlled by one or a few persons), the group scored low. This indicates that the group members are able to agree with each other, and follow others lead. Therefore does the group probably have the property "Dependence" \cite{Maturity}.

On statement 4 (...the work load is evenly distributed among the group members). This suggests that the group have the ability to control its workflow and distribution. The group also scored low on statement 6 (...the group shows little respect for regulations, to show up in time, keep appointments, prepare for or complete tasks effective and thoroughly). This can indicate that the group is structured and respect the authority of the other group members, which supports the view that group has the property "Control". Therefore we may assume that the maturity of the group is at level 3 (Production)\cite{Maturity}.

At statement 5 (...the group is facing internal opposition, disagreement/ ill will) the group scored low. This shows that the group does not have the property "Opposition". This means that it is a nice group to participate in, which is reflected in statement 3, however the group has not yet evolved to level 4 (Innovation)\cite{Maturity}.
 
The maturity level needed for the task is a bit more difficult to decide. The task is quite wide. It covers several fields of studies; hardware, software, Energy and power calculations, radio communication, and simulations. That means that we each focus on a specific part of the task, and are very dependent of each other, since no one can cover others part. Therefore it is reasonable to assume that the group need a leader that is able to make everything fit together. Like the operation allegory in the note. We may make the conclusion that the task needs a maturity level of 1 or 2. Since the group is not making anything new and innovative, it does not require a level 4 group. The task is not directly a production-oriented task either. The group is to produce one example, not a large quanta that requires a high level of control \cite{Maturity}.

As an additional point, the group task demanded a lot of assistance from external resources. When these resources were unable to aid us, the entire project halted, and caused us to be very ineffective. It would have been a huge advantage if an external coordinator made sure we could interact with the external resources when we needed it. This supports that this task demand a level 2 leadership structure.

After the group completed this exercise, we questioned our leader structure. This was a so-called "Undiscussable issue" to us. Since we have a nice social situation in the group. With the leadership structure we already had nobody needed to be more responsibly than the others, and nobody bossed the others about. However the Ground rules for effective teamwork clearly states that such undiscussable issues should be discussed\cite{EffectiveGroups}. This was difficult to discuss for us, however we reached some results. We found out that Eirik and Leif-Einar may have been a bit more eager to push our project forward previously, while Børge had been a more leader figure on the process side. With this in mind we tried the arrangement of a “one day leader”. 

We did see the effect of this action already the next week. The 10. of April, when we were going to write the report, Børge took charge. He divided the report up into several parts and wrote them on the blackboard. He gave everybody responsibility for different parts. Whenever a part was done, he crossed it out on the blackboard ad distributed a different part. By doing this, the group worked effectively in parallel the whole day. 

This shows how the awareness of our maturity level can make the group do necessary changes to complete a given task. If there exists a mismatch between the groups maturity level and the level needed by the task. If a group has a too low maturity level, it can be increased with Team Building exercises, which forces the group to move out of its comfort zone. In our case we believe the opposite situation was present, and this we solved by changing our leadership structure. It was a pity that this happened at the end of the project. We did however benefit from it the remaining time, this in the report writing phase of the project. The group members agreed that we had learned something from this result. When working in a group a leader should be reflected on at an early stage sine this might strengthen the group’s effectiveness. 


