\section{February 13, Changing project definition}

During our project we met some obstacles which made us change our project definition. We want to emphasis this day since we feel it represent how our group handled this set-back.

When we started our project we had decided on one type of ground station network that we wanted to test and use. We had spent time exploiting the BlueBox network and research was done by every group member. We learned that we were dependent on external partners to complete and work with our project, but were optimistic that we would get the information we needed. We where surprised when we got the news that Aalborg University did not have the opportunity to provide us with the needed information about BlueBox. Hallstein and Leif-Arne who had put in a lot of time understanding some of the technical aspects of BlueBox, felt that they had worked in vain and lost their motivation. As Weeland describes the fundamental attribution error \cite{EffectiveTeamMembers} we started blaming others for our defeat. Thoughts along the lines: "Its the Aalborg Universities fault that we have to start from the beginning, why would they not give us any help?" and " Its not our fault!" surfaced among different team members. 

As the motivation went down we all became ineffective and unproductive. But putting the blame on others would not help us get any further with our project. We took Weelands advice and started finding the factors that was blocking the group progress. By not blaming "the other guy", and remembering that all group members have responsibility for group success and failure got us more work willing. 

As a group we wanted to focus on creating an effective group, high-performance group, by following Schwartz ground rule nine \cite{EffectiveGroups}, which details decision making. We had earlier decided to have a flat structure within the group, without any appointed leader, and a democratic decision-making process \cite{EffectiveGroups}. This time we would ensure that the project was feasible, and used ground rule two. This rule stated that each member share all the relevant information she or he has that affects how the group solves a problem or makes a decision. We then put up a checklist over other ground station network that could be a solution for us. Every group member gathered information about one of the network that could be appropriate for the project. We then conducted a round where each member presented what they had found, and shared their views. We then discussed which alternative would suit the group best, and then according to our "Teamwork agreement" voted. The voting were unanimous as there were huge benefits by choosing one option, and we felt that we reached consensus since we all were well informed before the voting. Changing the project definition led to a discussion of delegation of tasks. 

With a group consisting of six people with different qualities and interests we had to discuss how we wanted to cooperate to solve our new project. First each member presented his ability and knowledge that could be of useful to reach our goal. Though we are a group consisting of different backgrounds we were determined to use this as an advantage. As Johnson and Johnson wrote \cite{ValuingDiversity}, "Tomorrow`s effective groups (including large groups such as organizations and nations) will be those that have learned to be productive with a diverse membership" . This led to ground rule eight in \cite{EffectiveGroups} where we had to discuss an undiscussable issue. Some of the group members felt at ease taking on different work tasks. Hanne whose study program differences the most from the core of the project, was a bit skeptical to how her competence could be used. She had to confess to the group her theoretical weaknesses that made her unsuitable for some tasks. To solve this we took action and shaped the project so that every member got an assignment where they could use their expertise. In this way everyone felt that they played a significant role in reaching our common goal. 