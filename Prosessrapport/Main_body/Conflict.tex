\section{March 13, Solving a conflict}


During the work period the group has been poor to start at the agreed time. In our group agreement it is written that if you come late to the agreed time the person must inform the group. This point has so far been adhered by all members, but it does not prevent people form being late. As one member of the group has consistently been late to group meetings the rest of the group decided to take this up. Even though he follows the group agreement with giving notice, he breaks the agreement of being on time. As this has been happening over time the rest of the group felt that it affected the working progress. Either the group waited until all the members came and lost working time, or they had to use time on getting the person concerned up to speed. Either way the situation would be better of, by discussing this with the person concerned.

As the group planned to bring it up, we noticed that our flat group structure made it difficult. Nobody wanted to be the "bad guy", but without any announced leader we had come up with a solution. As we did not want to be a group that avoided conflicts to help a member save face, we wanted to be an effective group. As our group had build trust and a sense of maturity we agreed that each group member would be strong enough to take negative feedback directly KILDE. To find a way to solve this problem on a constructively manner we followed one of the three effectiveness criteria’s written by Schwarz (p23). Schwarz (kilde) categorize problem solving as one of the primary group processes an effective group should manage. We wanted to use a systematic approach to the problem and find a solution after hearing and understanding the cause. Hanne took on the role as leader for the conflict handling. As Hallstein, the group member concerned, finally showed up, we sat him down to discuss the issue. On behalf of the group Hanne then told him how the group felt about the situation and how it affected the work process and collaboration. Hallstein felt that he had a lot of other schoolwork to deal with, but understood that he had to prioritize the group. He also stated that the group used to be unproductive at the beginning of the day, ant that this was not motivating when he hat other things to do. By taking this conflict up for discussion the group now knew what Hallstein felt and what the problem really was. To prevent further members from being late the group wrote a new point in the groupware agreement. We wanted to form a rule that would punishment the member that broke it but would work as an advantage for the others. It resulted in a cake rule; if one member was late he or she had to bring a cake to the next meeting. 

By identifying the cause of the problem, evaluating and selecting a solution the group felt that this conflict had contributed to personal growth. Hallstein had pointed out a problem with the groups effectively that the rest had not though of as the cause of the problem. 



Notater:
identify the cause of the problem, evaluate and select a solution, implement it and evaluate it. (p 23) 

