\section{March 13, Solving a conflict}

During the work period the group has had difficulties being productive at the start of the day. In our Teamwork agreement it is written that if someone is going to be late to the agreed upon time, that person must inform the rest of the group. This point has so far been adhered by all members, but it does not prevent people being late. As one member of the group has been consistently  late to group meetings the rest of the group decided to bring it up for discussion. Even though he follows the group agreement with giving notice, he breaks the agreement of being on time. As this has been happening over some time the rest of the group felt that it affected the project's progress. Either the group waited until all the members came and lost (valuable) time, or they had to use time on getting the person concerned up to speed. Either way the situation would hopefully be improved by discussing it with the person concerned.

As the group planned to bring it up, we noticed that our flat group structure made it difficult. Nobody wanted to be the "bad guy", but without any formal leader we had come up with a solution. We did not want to be a group that avoided conflicts to help a member save face, we wanted to be an effective group. As our group had grown to trust each other and we felt that every member was mature enough to not feel insulted by constructive feedback we decided to discuss the issue in a meeting. To find a way to solve this problem in a constructive manner we followed one of the three effectiveness criteria written by Schwarz\cite{WorkGroups}. Schwarz categorizes problem solving as one of the primary group processes an effective group should manage. We wanted to use a systematic approach to the problem and find a solution after hearing and understanding the cause. Hanne took on the role as leader for the conflict handling. All six members of the group sat down together and discussed the issue. On behalf of the group Hanne then told him how the group felt about the situation and how it affected the work process and collaboration. Hallstein felt that he had a lot of other schoolwork to deal with, but understood that he had to prioritize the group. He also stated that the group used to be unproductive at the beginning of the day, and that this was not motivating when he had other things to do. By taking this conflict up for discussion the group now knew what Hallstein felt and what the problem really was. To deter members from being late the group appended a new point to the Teamwork agreement. We wanted a rule that would punish the member that broke it, but would work as an advantage for the others. It resulted in a cake rule; if one member was late he or she had to bring a cake to the next meeting. 

By identifying the cause of the problem, evaluating and selecting a solution the group felt that this conflict had contributed to personal growth. Hallstein had pointed out a problem with the groups effectively that the rest had not thought of as the cause of the problem. Revealing this made the group reconsider their working methods. What could be done to secure more productivity at the start of the day? The teamwork agreement stated that we should start each day with a meeting where we updated each other on our progress and plan the rest of the day. The group has not been good at fulfilling this rule, which is one of the factors contributing to the unproductively. Børge suggested that we looked over the status of the project and located the remaining work. This made it easy to see what we had left to do and divide the work to increase production. Hallstein felt that by having more well defined tasks he would feel more motivated and committed to the group and the project.  

Already the next meeting we could see that the group worked more effectively. Everyone met at the given time and communication was improved. The group felt at ease since the conflict was resolved and everyone had said what they felt. We agreed that if someone felt that they were unsatisfied with the group or working methods, they should feel that they could take it up for discussion. It is also important that every member should be more critical of our own productivity and focus on continously improving the teamwork process.  
