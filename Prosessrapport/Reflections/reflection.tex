%\chapter{Group reflection}
%\label{chap:reflection}
%%\section{Personal reflections}
%%In this section we will each write down our reflections on the EiT subject and the team exercises we have discussed in this process report.
%%\subsection{Eirik}
At the start of this process report I wrote that I thought the EiT subject would mostly be formalizing things I already knew. I wasn't very positive to the course based on things I'd heard beforehand. When we're now approaching the end of the project period I have discovered that there are many interesting aspects to EiT, and while it does actually formalize alot of things, these are thing I wouldn't think about without this course. Specifically this is the importance of the group dynamics and how decisions are made, and not just the groups total competence, when it comes to the final product of the project. I have become more aware of my own role inside a team, and how the way I act affects the group, and also what the others does and doesn't do, and how this affects the group. I have become better at giving personal reflections to my group members and about my group members.

I also feel this group has worked exceptionally well together, maybe even too well, considering we had no major conflicts, and agreed on mostly all decisions. 

Personally I found the lectures on group psychology very interesting, and while the conflict part of these lectures didn't apply very much to our group, I recognized elements from other groups I've been part of, which really helped me realize the importance of knowing about this a bit more formally.
%At the end of the project the group sat down to reflect on the experiences and teamwork done during the project. We all came into the project with different expectations and academic background. We had all heard different things about the course, but we all had the impression that there would be potential conflicts and a lot of team exercises. The group had a wide variety in terms of academic background, and the project we chose to do allowed for using the skills of everyone on the team, and we had good communication and discussions where people showed their point of view, with no one being too quiet. We also feel that we allowed for everyone to use their skills and knowledge, and that we worked well together interdisciplinary.
%
%The group had a good social tone from the beginning, and everybody agreed on most decisions, or we discussed it and reached a decision. This meant that the group had fun during the project, even though we didn't get any larger conflicts we had some challenging situations. Our first group challenge came when we had to change our project definition and then we had a conflict with people not meeting at time. These events made the group more aware of how different theory can be used to solve group challenges. By openly discussing topics that are hard and reaching a common solution, we experienced a growth within the group. The group becomes more effective and the communication was improved.
%
%During the project we had a lot of team exercises. The team exercises first and foremost made us more aware of how we worked in a group, and how our behavior and level of involvement in the project affected the group. It also formalized quite a lot around conflicts and group roles, which became useful in this project. The Sosiogram we got from one of the facilitators was a good reflection on how we communicated at the start of the project. This made us more aware of how easy it is to forget the importance of everyones participation, and how it affects the group. To exploit the interdisciplinary, it is important that everyone participates in discussions. 
%
%As we look back at the working process we all felt that we have gained valuable experience. When communicating with people that differ from yourself it is important to present all valuable information in an understandable way. This helped our group a lot throughout the process since we did not share the same theoretical background. Further we learned that feedback is essential for the different team members to grow. By discussing how we see each other in the team, we all had something to work on, and will take with us valuable experience from this project.  


\chapter{Summary}
\label{chap:reflection}

WORK IN PROGRESS!!
%\section{Personal reflections}
%In this section we will each write down our reflections on the EiT subject and the team exercises we have discussed in this process report.
%\subsection{Eirik}
At the start of this process report I wrote that I thought the EiT subject would mostly be formalizing things I already knew. I wasn't very positive to the course based on things I'd heard beforehand. When we're now approaching the end of the project period I have discovered that there are many interesting aspects to EiT, and while it does actually formalize alot of things, these are thing I wouldn't think about without this course. Specifically this is the importance of the group dynamics and how decisions are made, and not just the groups total competence, when it comes to the final product of the project. I have become more aware of my own role inside a team, and how the way I act affects the group, and also what the others does and doesn't do, and how this affects the group. I have become better at giving personal reflections to my group members and about my group members.

I also feel this group has worked exceptionally well together, maybe even too well, considering we had no major conflicts, and agreed on mostly all decisions. 

Personally I found the lectures on group psychology very interesting, and while the conflict part of these lectures didn't apply very much to our group, I recognized elements from other groups I've been part of, which really helped me realize the importance of knowing about this a bit more formally.
At the end of the project the group sat down together to reflect on the experiences and teamwork done during the project, and also had a short meeting with the village leader about the same subject. We had all heard different things about the course and we all had the impression that there would be potential conflicts and a lot of team exercises. The group had a wide variety in terms of academic background, and the project we chose to do required using the skills of everyone on the team.

During this term there have been a lot of team exercises. They have first and foremost been helpful for facilitating discussions about how we work as a group and how our behavior and level of involvement in the project affected the group. It also formalized quite a lot around conflicts and groups rules, which became useful later in the project. 

The sosiogram we got from one of the facilitators was a good for reflection on how we communicated at the start of the project. In a interdisciplinary group the different people have different viewpoints and ideally the interaction between different fields will yield ideas etc. that would not be possible for a group with a narrower focus. This made us more aware of how easy it is to forget the importance of everyone’s participation, and how it affects the group. To exploit the interdisciplinary nature of our group, it was important that everyone participates in discussions. 

We had a good communication flow and discussions where people showed their point of view, with no one being too quiet. We also feel that we allowed for everyone to use their skills and knowledge, and that we worked well together interdisciplinary.

The group had a open and friendly tone from the beginning, and everybody agreed on most decisions, or we discussed it and reached a unanimous decision. Even though we didn't get any larger conflicts we had some challenging situations. Our first group challenge came when we had to change our project definition and then we had a conflict with people not meeting at time. These events made the group more aware of how different theory can be used to solve group challenges. By openly discussing topics that are hard and reaching a common solution, we experienced a growth within the group. The group became more effective and the communication flow improved.

As we look back at the working process we all felt that we have gained valuable experience. When communicating with people that differ from yourself it is important to present all valuable information in an understandable way. This helped our group a lot throughout the process since we did not share the same theoretical background. Further we learned that feedback is essential for the different team members to grow. By discussing how we see each other in the team, we all had something to work on.

We have also learned a lot about different models and characteristics for high performance groups which we feels

As we look back at our process we feel that we've gained valuable experience. a lot about different models for groups and the theoretical needs of high performance groups. There have also been situations that have highlighted that the extremely flat structure may not have been optimal. This was exacerbated by the nature of our project where we don't really need to answer to anyone else when it comes to progress etc. Which is in sharp contrast to most situations we will meet in "real life".

We also think we have done some things right, e.g. had a very open and friendly atmosphere and meetings without any distracting elements like computers. 

