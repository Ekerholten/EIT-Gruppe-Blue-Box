\begin{abstract}

The subject Experts in Teamwork builds on learning about interdisciplinary team collaboration. Our group consists of six people with different backgrounds. We will categorize us as a heterogeneous group despite that we all study at NTNU Gløshaugen. To use the different knowledge within the group we had a strong focus on shaping the task so that everyone can participate with their knowledge. Since no one had great background knowledge of the project's theme, it was natural for us to have a flat structure without any designated leader. 

To identify the group development and progress, various events and process exercises has been discussed and evaluated. As a team we have solved conflicts, reached consensus when changing the project definition and achieved better self-awareness and openness through process exercises. When we met challenges where the group had difficulties to cooperate or were unproductive, we used different theory to solve this in a good way.  
After working together for a while we got a good friendship that led to more openness and better communication. We also had a lot of fun during the working period. 

From this experience we all learned something about us selves and how we affect the group dynamic.  We also experienced how to cope with challenges and discuss undiscussable issues. 
\end{abstract}

\renewcommand{\abstractname}{Preface}
\begin{abstract}
This report is a part of the result of the Experts in Teamwork village TFE4850 Student Satellite, spring 2013. EiT is a interdisciplinary course, where students from different study programs work together to achieve a common goal. It is important to use the different knowledge in constructive manner and work well together under pressure to achieve effectiveness and reach our goal. 

An important part of this course is to focus on how the group work together and communicates internally. This is a report that describes this process, and we emphasize individual situations which we believe has been of great importance to the group collaboration. We have focused on using relevant theory and joint reflection to create an understanding of what happened,  what we could have done differently and how to improve the process on this basis. We will also try to locate patterns of behavior that affects the group work. 
\end{abstract}

