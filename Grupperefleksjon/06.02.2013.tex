% !TEX TS-program = pdflatex
% !TEX encoding = UTF-8 Unicode

% This is a simple template for a LaTeX document using the "article" class.
% See "book", "report", "letter" for other types of document.

\documentclass[11pt]{article} % use larger type; default would be 10pt

\usepackage[utf8]{inputenc} % set input encoding (not needed with XeLaTeX)

%%% Examples of Article customizations
% These packages are optional, depending whether you want the features they provide.
% See the LaTeX Companion or other references for full information.

%%% PAGE DIMENSIONS
\usepackage{geometry} % to change the page dimensions
\geometry{a4paper} % or letterpaper (US) or a5paper or....
% \geometry{margin=2in} % for example, change the margins to 2 inches all round
% \geometry{landscape} % set up the page for landscape
%   read geometry.pdf for detailed page layout information

\usepackage{graphicx} % support the \includegraphics command and options

% \usepackage[parfill]{parskip} % Activate to begin paragraphs with an empty line rather than an indent

%%% PACKAGES
\usepackage{booktabs} % for much better looking tables
\usepackage{array} % for better arrays (eg matrices) in maths
\usepackage{paralist} % very flexible & customisable lists (eg. enumerate/itemize, etc.)
\usepackage{verbatim} % adds environment for commenting out blocks of text & for better verbatim
\usepackage{subfig} % make it possible to include more than one captioned figure/table in a single float
% These packages are all incorporated in the memoir class to one degree or another...

%%% HEADERS & FOOTERS
\usepackage{fancyhdr} % This should be set AFTER setting up the page geometry
\pagestyle{fancy} % options: empty , plain , fancy
\renewcommand{\headrulewidth}{0pt} % customise the layout...
\lhead{}\chead{}\rhead{}
\lfoot{}\cfoot{\thepage}\rfoot{}

%%% SECTION TITLE APPEARANCE
\usepackage{sectsty}
\allsectionsfont{\sffamily\mdseries\upshape} % (See the fntguide.pdf for font help)
% (This matches ConTeXt defaults)

%%% ToC (table of contents) APPEARANCE
\usepackage[nottoc,notlof,notlot]{tocbibind} % Put the bibliography in the ToC
\usepackage[titles,subfigure]{tocloft} % Alter the style of the Table of Contents
\renewcommand{\cftsecfont}{\rmfamily\mdseries\upshape}
\renewcommand{\cftsecpagefont}{\rmfamily\mdseries\upshape} % No bold!

%%% END Article customizations

%%% The "real" document content comes below...

\title{Grupperefleksjon}
\author{Gruppe 2}
%\date{} % Activate to display a given date or no date (if empty),
         % otherwise the current date is printed 

\begin{document}
\maketitle
Gruppen følte at vi var litt lite produktive og at vi måtte komme igang. 
Vi diskuterte hvordan vi kunne effektivisere arbeidsprosessen vår og fant ut at vi kunne pararellisere arbeidet.  Gruppen delte seg i 2. Den ene
gruppen skulle jobbe med å lage en liste over hvilken informasjon vi trenger fra
danskene. Den andre gruppen skulle finpusse prosjektdefenisjonen samt fullføre arbeidsplanen. 
Dette fungerte bra og vi følte at vi økte effektiviteten ved å pararellisere arbeidet. 
Det vi kan lære av denne hendelsen til neste gang er at vi kan identifisere inneffektivitet før dagen flyr fra oss.

Dagen ble lite produktiv fordi Roger var syk/bortreist. Vi manglet derfor endel av den informasjonen vi trengte for å lage en god prosjektdefenisjon. Vi fikk tak i den informasjonen som var tilgjengelig på nett, men vi følte at dette ikke helt var tilstrekkelig. For videre læring burde vi kanskje sjekket status før landsbydagen slik at vi kunne vært litt bedre forberedt. 

To stykk kom 10 minutter for sent på grunn av forsinket buss. Dette gikk greit denne gangen da vi bare så vidt var kommet i gang med innsjekken. Vi kom til at vi må vite om folk er litt forsinket eller ikke kommer den dagen. For å få den informasjonen legger vi ut telefonnummer på itslearning sånn at man kan bli kontaktet om nødvendig.
\end{document}
