% !TEX TS-program = pdflatex
% !TEX encoding = UTF-8 Unicode

% This is a simple template for a LaTeX document using the "article" class.
% See "book", "report", "letter" for other types of document.

\documentclass[11pt]{article} % use larger type; default would be 10pt

\usepackage[utf8]{inputenc} % set input encoding (not needed with XeLaTeX)

%%% Examples of Article customizations
% These packages are optional, depending whether you want the features they provide.
% See the LaTeX Companion or other references for full information.

%%% PAGE DIMENSIONS
\usepackage{geometry} % to change the page dimensions
\geometry{a4paper} % or letterpaper (US) or a5paper or....
% \geometry{margin=2in} % for example, change the margins to 2 inches all round
% \geometry{landscape} % set up the page for landscape
%   read geometry.pdf for detailed page layout information

\usepackage{graphicx} % support the \includegraphics command and options

% \usepackage[parfill]{parskip} % Activate to begin paragraphs with an empty line rather than an indent

%%% PACKAGES
\usepackage{booktabs} % for much better looking tables
\usepackage{array} % for better arrays (eg matrices) in maths
\usepackage{paralist} % very flexible & customisable lists (eg. enumerate/itemize, etc.)
\usepackage{verbatim} % adds environment for commenting out blocks of text & for better verbatim
\usepackage{subfig} % make it possible to include more than one captioned figure/table in a single float
% These packages are all incorporated in the memoir class to one degree or another...
\usepackage{parskip}

%%% HEADERS & FOOTERS
\usepackage{fancyhdr} % This should be set AFTER setting up the page geometry
\pagestyle{fancy} % options: empty , plain , fancy
\renewcommand{\headrulewidth}{0pt} % customise the layout...
\lhead{}\chead{}\rhead{}
\lfoot{}\cfoot{\thepage}\rfoot{}

%%% SECTION TITLE APPEARANCE
\usepackage{sectsty}
\allsectionsfont{\sffamily\mdseries\upshape} % (See the fntguide.pdf for font help)
% (This matches ConTeXt defaults)

%%% ToC (table of contents) APPEARANCE
\usepackage[nottoc,notlof,notlot]{tocbibind} % Put the bibliography in the ToC
\usepackage[titles,subfigure]{tocloft} % Alter the style of the Table of Contents
\renewcommand{\cftsecfont}{\rmfamily\mdseries\upshape}
\renewcommand{\cftsecpagefont}{\rmfamily\mdseries\upshape} % No bold!

%%% END Article customizations

%%% The "real" document content comes below...

\title{Grupperefleksjon}
\author{Gruppe 2}
%\date{} % Activate to display a given date or no date (if empty),
         % otherwise the current date is printed 

\begin{document}
\maketitle 
I dag hadde alle gruppene en presentasjon av sin prosjektoppgave, mens de andre gruppene ga tilbakemelding og stilte spørsmål. Under denne seansen fikk vi identifisert styrker, svakheter og mulighet for vidreutvikling av vårt prosjekt. En viktig svakhet var at vi er svært avhengige av kontakt og bistand fra Aalborg universitet. Med dette i bakhodet tok vi en samtale med vår veileder Roger. Han foreslo at vi skulle undersøke alternative typer "Nettverksbokser". I tillegg kan vi prøve å simulere et bakkenettverk for å finne ut hvor mange bakkestasjoner vi trenger, og hvor disse burde være plasssert for å overføre mest mulig data uten at satelitten går tom for strøm. Derfor bestemte vi oss for å undersøke boksløsningene GENSO, Pico, Carpcomm, samt satelitt-simuleringsverktøyet STK \\

Samtalen med Roger ble veldig aktuell etter lunsj. Da hadde vi fått tilbakemelding fra Aalborg universitet. De skulle skyte opp en satelitt om to uker, og hadde ikke kapasitet til å bistå oss. Siden den nåværende versjonen av Blue Box hadde så mange mangler, annbefalte de oss å ikke bygge den, men heller vente på neste versjon. Da denne versjonen trolig ikke er ferdig i løpet av vår prosjektperiode blir det nok lite aktuelt å bygge blue box. Dette betyr i praksis at vi blir nødt til å skifte problemstilling til prosjektet. Alle gruppemedlemmene var innstilt på å lage en faktisk dings, framfor et teoretisk case study. Derfor mistet vi litt motivasjonen og ble litt uproduktive en liten periode. Likevel leste vi oss opp GENSO, Pico og Carpcomm. Da fant vi ut at Carpcomm allerede har ett oppegående nettverk med mange noder, og selger en ferdig bygget boks, som tilsvarer Blue Box. Vi diskuterte mulighetene for å kjøpe en slik boks for å koble den opp med antennen på taket. Vår veileder var positiv, men vi må først forhøre oss med landsbyens borgemester Bjørn B. Larsen siden denne boksen var litt dyr.  \\

Selv om dette har vært en dag med dårlige nyheter, har gruppen stort vært ved godt mot og jobbet effektivt. Leif Einar og Hallstein har brukt mye tid og energi på å få oversikt over kretsskjemaet til Blue Box. Derfor var disse to litt motvillig til å gå bort fra den orginale problemstillingen. Men begge to innser at det ikke kommer til å bli realistisk å bygge en vår egne boks, derfor er de innstilt på gjøre en like god jobb med å sette seg inn i Carpcomm. \\

Marius har det veldig travelt på jobb for tiden. Derfor var han nødt til å dra på jobb klokken halv tolv i dag. Resten av gruppen har forståelse for dette, og lot ham dra uten noen sure miner.   
\end{document}
