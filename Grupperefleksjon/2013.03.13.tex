\documentclass[10pt,a4paper]{article}
\usepackage[utf8]{inputenc}
\usepackage{amsmath}
\usepackage{amsfonts}
\usepackage{amssymb}

\title{Grupperefleksjon}

\author{Gruppe 2}

\begin{document}
\maketitle

Vi startet dagen med at vi satte opp en dagsplan, som samsvarer med at vi ville følge sammarbeidsplanen bedre, da dette var et av punktene vi ikke hadde gjort godt nok. Vi startet med dette fulle av entusiasme, men vi husket ikke og ta kontakt med radio-ansvarlig, så dette førte til at vi ikke fikk gjennomført planen, da vedkommende ikke var her. Dette førte til at luften gikk litt ut av gruppa, og vi var enige om at et passende tiltak er at vi i fremtiden må passe på at de vi er avhengige av er der, når planen avhenger av dem. Resten av dagen fulgte vi plan B, som var å skrive mer på rapporten.

Hallstein forsov seg igjen, og da det står i sammarbeidsavtalen at ved gjentatte forseelser skulle dette tas opp, tok vi opp dette, og alle parter var enige om at han skulle bli bedre på dette. På grunn av dette ble vi også enige om at hvis man ikke møter opp eller sier fra i tide, må man ta med seg kake den dagen eller neste (hvis man ikke kommer i det hele tatt). Dette var den første tingen vi tok opp i gruppen som gikk på personlig forseelse, og alle syntes det var bra at dette gikk greit.
%Vi registrerer at vi ikke har vært beviste på sammarbeidsplanen. Vi har bestemt at alle skal være mer bevist på sammarbeidsplanen %og eventuelt revurdere den senere.

%Vi hadde fremdeles problemer med pc-er under møte. Etter at vi ble beviste på dette la vi fra oss dataene og gjennomførte et %effektivt møte. Vi har kommet til at vi skal følge sammarbeidsplanen og skal ha en ordstyrer som sier fra at nå skal vi legge fra oss %datamaskinene og ha møte.

%Etter at vi fikk satt opp Raspberry pi og instalert programvaren følte vi at nå har vi kommet skikkelig i gang med prosjektet.

%Det var bra at vi delte oss i flere grupper som jobbet med hvert sitt delprosjekt. Det er også bra at alle allikevel vet hva alle driver %med. Vi hadde også et eget møte på slutten av dagen for at alle skulle vite hva alle andre hadde utrettet i løpet av dagen.

%Vi føler at dagen blir mye oppstyket ved at vi må jobbe med prossesting midt på dagen. Dette gjør at vi aldri rekker å komme %skikkelig igang med arbeidet. Det blir bedre fremover siden vi vet når vi skal jobbe med prossesting. Vi skal også snakke med %fasilitatorene for å prøve å få samlet alle prosessoppgaver på starten av dagen.
\end{document}