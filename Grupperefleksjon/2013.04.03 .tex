\documentclass[10pt,a4paper]{article}
\usepackage[utf8]{inputenc}
\usepackage{amsmath}
\usepackage{amsfonts}
\usepackage{amssymb}

\title{Grupperefleksjon}

\author{Gruppe 2}

\begin{document}
\maketitle

Selv om kakesanksjoner ble innført som en regel forrige landsbydag, var det likevel kun tre som møtte opp til tide i dag. Dette var fordi en person var syk, en annen person kom med nattog etter ekskursjon til Japan og en person trodde det var tirsdag. Alle deltagerene prøvde å møte opp til tide, men det var unormale omstendigheter som førte til at enkelte ikke møtte til tide. Dette betyr at det blir vanskelig å bestemme om kaketiltaket har innvirkning på oppmøte.   

Under dagens prosessøvelse diskuterte vi gruppens egenskaper og dimensjoner i lys av Endre Sjøvolds notat "Maturity and effectiveness in small groups". Vi kom fram til at en annen organisering av gruppestrukturen kan ha en positiv innvirkning på vår effektivitet. Prosessøvelsen viser også at gruppen ikke går utenfor sin komfortsone, noe som hindrer at gruppens modenhet utvikler seg. Symptomer på dette er at vi ikke er i opposisjon eller stiller hverandre kritiske spørsmål.

Dersom vi er mer bevist på å stille hverandre kritiske spørsmål, kan vi oppnå to ting. Det ene er at vi ender opp med mer opposisjon (av den gode typen). Det andre er at vi kommer litt utenfor vår komforsone når vi må argumentere mer for våre synspunkter.


I dag har vi sett filmen av midtveispresentasjonen. Der kom det fram at vi kunne trenge en generalprøve før framføring for å samkjøres. I tillegg virket det som om Hallstein kanskje pratet litt lavt. Før vi presenterte dagens teori hadde vi en liten gjennomgang av presentasjonen, i tillegg fokuserte Hallstein på å prate høyere. Med disse tiltakene følte vi selv at vi hadde en mer koordinert og hørbar presentasjon. 
 


\end{document}