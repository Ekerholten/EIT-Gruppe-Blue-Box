\documentclass[10pt,a4paper]{article}
\usepackage[utf8]{inputenc}
\usepackage{amsmath}
\usepackage{amsfonts}
\usepackage{amssymb}

\title{Grupperefleksjon}

\author{Gruppe 2}

\begin{document}
\maketitle

Dagen har vært preget av rapportskriving. Det første som skjedde var at Børge tok ansvar og delte opp rapporten i deler, hvor alle fikk i oppgave å skrive hver sin del. Dette førte til at alle gruppedeltakerne har vært aktive hele dagen, og det har vært lite dødtid. Dette kan mulig bekrefte våre mistanker, om at en gruppekoordinator kan effektivisere vår arbeidsprosess.  

Etter lunsj hadde vi en prosessøvelse hvor vi skulle gi både positiv og konstruktiv tilbakemelding til de andre på gruppen. Noen opplevde denne øvelsen som vanskelig. Denne øvelsen var vanskelig siden vi måtte gi personlig tilbakemelding på en persons prestasjoner og personlighet under gruppearbeidet. Dette førte til at vi måtte aktivt oppsøke og diskutere en persons svakere sider, noe som er litt utenfor vår komfortsone. Ut ifra de refleksjonene vi gjorde oss forrige uke, er dette noe vi trenger. Når vi har det ukomfortabelt vokser vi i modenhet som gruppe. Nå vet alle deltakerne hva de andre på gruppen syns om en. Da kan vi jobbe med oss selv, for å bli bedre gruppedeltakere. I tillegg er det letter å stole på de andre i gruppen, når vi vet hva vi synes om hverandre. Selv om øvelsen var noe ukomfortabel mens den pågikk, tror vi at den har positiv effekt på vår utvikling som gruppe.       

\end{document}